\chapter{Requirements}
\label{ch:requirements}

\section{Popularity}
As it should be possible for customers to write extensions to the IAM platform, the popularity of the language in which these should be written is of great importance. Customers should be able to easily find a developer who can write in the chosen language. As discussed in the literature review, the most popular server-side programming languages are as follows:
\begin{itemize}
    \item JavaScript
    \item Python
    \item Java
    \item TypeScript
    \item C++
    \item C\#
    \item C
    \item PHP
    \item Go
    \item Kotlin
    \item Rust
\end{itemize}

This list serves as the long list mentioned in the methodology chapter.

\section{Memory Safety}
As mentioned in the literature review, memory safety issues are an extremely common cause for security vulnerabilities. This would make it absolutely irresponsible to use a language that is not memory-safe in an environment so specifically built around cybersecurity. This decisively rules out C and C++ as possible options.

\section{Static Types}
% TODO

\section{Performance}
Considering these actions will be executed in response to network calls, and will typically make network calls (such as database queries and HTTP requests) themselves, the performance impact of the programming language choice will be negligible compared to the time the network calls take up. For this reason, performance is not a relevant requirement in this specific case.

\section{Readability}
% TODO

\section{Testing}
% TODO

\section{Final evaluation}
The table below lists each language in the long list and checks them against the aforementioned requirements, sorted by descending popularity percentage from the literature review.
\begin{center}
\begin{tabular}{|l| c c c |}
    \hline
    Language & Popularity & Memory safe & Statically typed \\
    \hline
    JavaScript & 62\% & Yes & No \\
    Python     & 52\% & Yes & No \\
    Java       & 40\% & Yes & Yes \\
    TypeScript & 36\% & Yes & Yes \\
    C++        & 24\% & No  & Yes \\
    C\#        & 24\% & Yes & Yes \\
    C          & 19\% & No  & Yes \\
    PHP        & 18\% & Yes & Certain aspects \\
    Go         & 15\% & Yes & Yes \\
    Kotlin     & 12\% & Yes & Yes \\
    Rust       & 12\% & Yes & Yes \\
    \hline
\end{tabular}
\end{center}

Based on this evaluation, TypeScript and Java appear to be the most promising options. With TypeScript and Java being the two languages currently used internally at TrustBuilder, this selection is especially interesting, as this means the necessary expertise in these languages will already present at the company. 

In the following chapter, a proof of concept will be made in each of these languages to further compare them and their suitability.
