\chapter{Proof of Concept}
\label{ch:proof-of-concept}
To evaluate the suitability of the languages in the short list, a proof of concept was made that covers some typical use cases of TrustBuilder's extensions. 

These extensions will made up of units called Actions, which can each handle a specific task. Actions can take inputs and produce outputs, and be tied together into a workflow. The way these actions are composed into workflows will be the responsibility of the encompassing framework, and is out of scope for this thesis.

The proof of concept for each language will consist of the following:
\begin{itemize}
    \item An Action that logs a statement based on input, producing a side effect but no output
    \item An Action that retrieves user information from a REST API. The GitHub API is used here as an example.
    \item An Action that retrieves user information from a database. A local SQLite database is used here as an example.
\end{itemize}

\section{Implementation}
\subsection{Java}

\begin{listing}[H]
\begin{minted}{java}
public interface Action<I, O> {
    O execute(I input);
}
\end{minted}
\caption{Action interface}
\end{listing}

\begin{listing}[H]
\begin{minted}{java}
public class LogAction implements Action<LogAction.Input, Void> {
    private final Logger logger;
    
    public LogAction() {
        logger = LoggerFactory.getLogger(LogAction.class);
    }
    
    public LogAction(Logger logger) {
        this.logger = logger;
    }
    
    @Override
    public Void execute(Input input) {
        logger.makeLoggingEventBuilder(input.level).log(input.message, input.params);
        return null;
    }
    
    public record Input(
        Level level,
        String message,
        Object... params
    ) {
    }
}
\end{minted}
\caption{An SLF4J logging Action}
\end{listing}

\begin{listing}[H]
\begin{minted}{java}
public class GetGitHubUserAction implements Action<GetGitHubUserAction.Input, GetGitHubUserAction.GitHubUser> {
    private static final URI GITHUB_USERS_URI = URI.create("https://api.github.com/users/");
    
    private final Gson gson = new GsonBuilder()
    .setFieldNamingPolicy(FieldNamingPolicy.LOWER_CASE_WITH_UNDERSCORES)
    .create();
    
    @Override
    public GitHubUser execute(Input input) {
        try (var client = HttpClient.newHttpClient()) {
            HttpRequest request = HttpRequest.newBuilder()
            .uri(GITHUB_USERS_URI.resolve(input.username))
            .build();
            String response = client.send(request, HttpResponse.BodyHandlers.ofString()).body();
            return gson.fromJson(response, GitHubUser.class);
        } catch (IOException e) {
            throw new ActionException("Error occurred trying to reach GitHub API", e);
        } catch (InterruptedException e) {
            Thread.currentThread().interrupt();
            throw new ActionException("Thread was interrupted while trying to reach GitHub API", e);
        }
    }
    
    public record Input(
        String username
    ) {
        
    }
    
    public record GitHubUser(
        String login,
        String type,
        String name,
        String company,
        String location,
        int publicRepos,
        int publicGists,
        int followers,
        int following
    ) {}
}
\end{minted}
\caption{A Java Action that retrieves a user from the GitHub API}
\end{listing}

\begin{listing}[H]
\begin{minted}{java}
public class GetDatabaseUserAction implements Action<GetDatabaseUserAction.Input, GetDatabaseUserAction.DatabaseUser> {
    
    private final Jdbi jdbi;
    
    public GetDatabaseUserAction(String url) {
        this.jdbi = Jdbi.create(url);
    }
    
    @Override
    public DatabaseUser execute(Input input) {
        return jdbi.withHandle(handle ->
            handle.createQuery("SELECT id, given_name, family_name, username, email FROM users WHERE username = :username")
                .bind("username", input.username)
                .map((rs, ctx) -> new DatabaseUser(
                    rs.getInt("id"),
                    rs.getString("given_name"),
                    rs.getString("family_name"),
                    rs.getString("username"),
                    rs.getString("email")
                )).one()
        );
    }
    
    public record Input(
        String username
    ) {
    }
    
    public record DatabaseUser(
        int id,
        String givenName,
        String familyName,
        String username,
        String email
    ) {
    }
}
\end{minted}
\caption{A Java Action that retrieves user data from a database}
\end{listing}

\subsection{TypeScript}

\begin{listing}[H]
\begin{minted}{typescript}
export type Action<Input, Output> = (input: Input) => Promise<Output>
\end{minted}
\caption{The Action type}
\end{listing}

\begin{listing}[H]
\begin{minted}{typescript}
export const logAction: Action<LogActionInput, void> = async (input) => {
    const logger = input.logger ?? winston.createLogger({
        transports: new winston.transports.Console()
    })
    
    logger.log(input.level, input.message)
}

interface LogActionInput {
    logger?: winston.Logger | null,
    level: LogLevel,
    message: string
}

type LogLevel = "silly" | "debug" | "verbose" | "http" | "info" | "warn" | "error"
\end{minted}
\caption{A Winston logging action}
\end{listing}

\begin{listing}[H]
\begin{minted}{typescript}
export const getGitHubUserAction: Action<GetGitHubUserActionInput, GithubUser> = async (input) => {
    const res = await fetch(`https://api.github.com/users/${input.username}`)
    
    return res.json();
}

interface GetGitHubUserActionInput {
    username: string
}

interface GithubUser {
    login: string,
    type: string,
    name: string,
    company: string,
    location: string,
    public_repos: number,
    public_gists: number,
    followers: number,
    following: number
}
\end{minted}
\caption{An TypeScript Action that retrieves a user from the GitHub API}
\end{listing}

\begin{listing}[H]
\begin{minted}{typescript}
const users = sqliteTable('users', {
    id: integer('id').primaryKey(),
    givenName: text("given_name"),
    familyName: text("family_name"),
    username: text("username").notNull(),
    email: text("email").notNull()
})

const db = drizzle(new Database('database.db'), {schema: {users: users}})

export const getDbUser: Action<GetDbUserActionInput, DbUser | undefined> = async (input) => {
    return await db.query.users.findFirst({
        where: eq(users.username, input.username)
    })
}

interface GetDbUserActionInput {
    username: string
}

interface DbUser {
    id: number,
    givenName?: string | null,
    familyName?: string | null,
    username: string,
    email: string
}
\end{minted}
\caption{A TypeScript Action that retrieves data from a database}
\end{listing}

\section{Findings}
\subsection{Java}
\subsection{TypeScript}
