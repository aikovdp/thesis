%%=============================================================================
%% Inleiding
%%=============================================================================

\chapter{\IfLanguageName{dutch}{Inleiding}{Introduction}}%
\label{ch:inleiding}

This bachelor thesis is written based on a business case from TrustBuilder, a European company providing Identity and Access Management (IAM) solutions for organizations across Europe, with extensive customization capabilities to satisfy client-specific needs.

Their product, also named TrustBuilder, is a platform that aims to be able to handle any IAM need, such as authentication, session management, authorization, and user management. Further requirements can take many forms, and the TrustBuilder platform needs to be able to tightly integrate with each client’s user-facing applications and other services. As these needs may differ from application to application, not everything can be a native part of the IAM system, and customization capabilities are a must. To satisfy these needs, their clients need to be able to to customize and extend their platform using custom code.

TrustBuilder tackles this problem by allowing for the creation of \emph{workflows}. These workflows consist of simple building blocks such as \emph{script blocks}, \emph{adapters} and \emph{services}, using arrow connectors and \emph{condition blocks} between them for control flow. Script blocks can run custom JavaScript functions, allowing workflows to be used to implement virtually any functionality missing from the base product, as well as interface with the base product directly.

\section{\IfLanguageName{dutch}{Probleemstelling}{Problem Statement}}%
\label{sec:probleemstelling}

TrustBuilder wants to overhaul their workflow platform to improve the developer experience and maintainability of workflows. Therefore, they want to know what language(s) their workflow platform should be built around. Currently, JavaScript is used to script these extensions, but this decision was made over a decade ago and needs to be re-evaluated.

\section{\IfLanguageName{dutch}{Onderzoeksvraag}{Research question}}%
\label{sec:onderzoeksvraag}

This thesis will investigate which languages would be the best choice for scripting extensions and customizations for the TrustBuilder IAM platforms. This encompasses the following sub-questions: 
\begin{itemize}
    \item{What are the key indicators these languages should be evaluated against?}
    \item{What would be the pros and cons of specific language choices?}
    \item{What are the security implications of specific language choices?}
\end{itemize}

\section{\IfLanguageName{dutch}{Onderzoeksdoelstelling}{Research objective}}%
\label{sec:onderzoeksdoelstelling}

A primary language choice based on the performed research and a proof of concept showcasing the language’s suitability for IAM-related tasks will allow TrustBuilder to start the overhaul of workflows with certainty.
This thesis aims to provide insight into the process behind that selection, the advantages and disadvantages of its use, and compare it to a secondary contender. Both the primary and secondary language choices will be put to the test in a proof of concept that covers typical use cases of TrustBuilder's current extension system.

\section{\IfLanguageName{dutch}{Opzet van deze bachelorproef}{Structure of this bachelor thesis}}%
\label{sec:opzet-bachelorproef}

The remainder of this bachelor thesis is structured as follows:

In chapter~\ref{ch:stand-van-zaken}, an overview of the state of the art of the research domain is given, based on a literature review.

In chapter~\ref{ch:methodologie}, the methodology is explained, and the research methods used to formulate an answer to the research question are discussed.

In chapter~\ref{ch:requirements}, the requirements of the final language choice are discussed. A long list of options is evaluated against these requirements to create a short list of two languages which will each be used in a proof of concept.

In chapter~\ref{ch:proof-of-concept}, each of the two languages in the short list are used to create a proof of concept which covers common use cases for TrustBuilder's extension system, and the findings in making these will be discussed.

Finally, in chapter ~\ref{ch:conclusie}, a conclusion and an answer to the research question is given, with a primary and secondary language choice and a summary of the reasons behind it.
