%==============================================================================
% Sjabloon onderzoeksvoorstel bachproef
%==============================================================================
% Gebaseerd op document class `hogent-article'
% zie <https://github.com/HoGentTIN/latex-hogent-article>

% Voor een voorstel in het Engels: voeg de documentclass-optie [english] toe.
% Let op: kan enkel na toestemming van de bachelorproefcoördinator!
\documentclass{hogent-article}

% Invoegen bibliografiebestand
\addbibresource{voorstel.bib}

% Informatie over de opleiding, het vak en soort opdracht
\studyprogramme{Professionele bachelor toegepaste informatica}
\course{Bachelorproef}
\assignmenttype{Onderzoeksvoorstel}
% Voor een voorstel in het Engels, haal de volgende 3 regels uit commentaar
% \studyprogramme{Bachelor of applied information technology}
% \course{Bachelor thesis}
% \assignmenttype{Research proposal}

\academicyear{2022-2023} % TODO: pas het academiejaar aan

% TODO: Werktitel
\title{Vul hier de voorgestelde titel van je onderzoek in}

% TODO: Studentnaam en emailadres invullen
\author{Ernst Aarden}
\email{ernst.aarden@student.hogent.be}

% TODO: Medestudent
% Gaat het om een bachelorproef in samenwerking met een student in een andere
% opleiding? Geef dan de naam en emailadres hier
% \author{Yasmine Alaoui (naam opleiding)}
% \email{yasmine.alaoui@student.hogent.be}

% TODO: Geef de co-promotor op
\supervisor[Co-promotor]{S. Beekman (Synalco, \href{mailto:sigrid.beekman@synalco.be}{sigrid.beekman@synalco.be})}

% Binnen welke specialisatierichting uit 3TI situeert dit onderzoek zich?
% Kies uit deze lijst:
%
% - Mobile \& Enterprise development
% - AI \& Data Engineering
% - Functional \& Business Analysis
% - System \& Network Administrator
% - Mainframe Expert
% - Als het onderzoek niet past binnen een van deze domeinen specifieer je deze
%   zelf
%
\specialisation{Mobile \& Enterprise development}
\keywords{Scheme, World Wide Web, $\lambda$-calculus}

\begin{document}

\begin{abstract}
  Hier schrijf je de samenvatting van je voorstel, als een doorlopende tekst van één paragraaf. Let op: dit is geen inleiding, maar een samenvattende tekst van heel je voorstel met inleiding (voorstelling, kaderen thema), probleemstelling en centrale onderzoeksvraag, onderzoeksdoelstelling (wat zie je als het concrete resultaat van je bachelorproef?), voorgestelde methodologie, verwachte resultaten en meerwaarde van dit onderzoek (wat heeft de doelgroep aan het resultaat?).
\end{abstract}

\tableofcontents

% De hoofdtekst van het voorstel zit in een apart bestand, zodat het makkelijk
% kan opgenomen worden in de bijlagen van de bachelorproef zelf.
%---------- Inleiding ---------------------------------------------------------

\section{Introduction}%
\label{sec:introduction}

This research proposal is written based on a business case of TrustBuilder, a European company providing Identity and Access Management (IAM) solutions for organizations across Europe, with extensive customization capabilities to satisfy client-specific needs.

Their product, also named TrustBuilder, is platform that aims to be able to handle any IAM need, such as authentication, session management, authorization, and user management. Further requirements can take many forms, and the TrustBuilder platform needs to be able to tightly integrate with each client’s user-facing applications and other services. As these needs may differ from application to application, not everything can be a native part of the IAM system, and customization capabilities are a must. In order to satisfy these needs, their clients need to be able to to customize and extend their platform using custom code.

To tackle this problem, TrustBuilder allows for the creation of “workflows”. These workflows consist of simple building blocks such as “script blocks”, “adapters” and “services”, using arrow connectors and “condition” blocks between them for control flow. Script blocks can run custom JavaScript functions, allowing workflows to be used to implement virtually any functionality missing from the base product, as well as integrate with the base product directly.

TrustBuilder wants to overhaul their workflow platform to improve the developer experience and maintainability of workflows. Therefore, they want to know what language(s) their workflow platform should be built around.

This thesis will investigate which languages would be the best choice for scripting extensions and customizations for IAM platforms. A primary language choice based on the performed research and a proof of concept showcasing the language’s suitability for IAM-related tasks will allow TrustBuilder to start the overhaul of workflows with certainty.

%---------- Stand van zaken ---------------------------------------------------

\section{State of the art}%
\label{sec:state-of-the-art}

\subsection{Comparable services}
\subsubsection{Auth0 Actions}
Auth0 is an IAM platform that enables developers to add authentication and authorization services to their applications \autocite{Auth0Overview}. Actions, one of their customization capabilities, let customers ``customize and extend Auth0's capabilities with custom logic'' \autocite{Auth0Actions}. Auth0 Actions are Node.JS JavaScript functions which can be executed at specific points in identity flows.

\subsubsection{Cloudflare Workers}
A more general-purpose example, Cloudflare Workers, announced by Kenton \textcite{Varda2017}, were created to allow their customers to run code on Cloudflare's edge network, to cover use cases Cloudflare themselves can't. A Cloudflare Worker could initially only be written in JavaScript, but can now run code written in any language that compiles to JavaScript (such as TypeScript, Python, and Kotlin) or WebAssembly (such as C, C++, Rust and Go). \autocite{Varda2018, Koeninger2020}

\subsection{Server-side languages}
To determine possibly suitable programming languages, surveys may provide useful data to determine the currently most popular ones. The Stack Overflow Developer Survey 2023 \autocite{StackOverflow2023} and The State of Developer Ecosystem 2023 \autocite{JetBrains2023} each provide statistics on programming language popularity. Based on these results, the four most popular Server-side programming languages are JavaScript, Python, Java, and TypeScript. Further popular languages include   C C++, C\#, PHP, Go, Kotlin, and Rust.

\subsubsection{JavaScript and TypeScript}
In both surveys, JavaScript proves to be the most widely used programming language across developers. Despite having originally been created for usage in browsers \autocite{NCC1995}, it can also be used in server-side environments on runtime systems such as Node.js \autocite{OpenJSFoundation}.

According to the TypeScript Handbook, TypeScript is a typed superset of JavaScript \autocite{TypeScript2023}. Its compiler checks code for type errors, and will then produce plain JavaScript code. This allows the compiled code to be run in any JS runtime environment.

\subsubsection{Python}
Python is a general-purpose programming language designed with simplicity and readability in mind \autocite{Peters2004}. Based on GitHub's 2022 Octoverse report, \textcite{Scarlett2023} states that Python is commonly used for web and software development, task automation, machine learning and data science, financial analysis, and artificial intelligence.

The official Python documentation describes it as being ``suitable as an extension language for customizable applications'' \autocite{PSF2023}.

\subsubsection{Java}
Java is a general-purpose object-oriented programming language \autocite{Lindholm2015}. A Java application may be compiled to instructions for the Java Virtual Machine (JVM), which allows the language to be hardware- and operating system-independent. Other popular JVM languages include Kotlin and Scala \autocite{StackOverflow2023, JetBrains2023}.

\subsubsection{WebAssembly}
As stated on its home page, ``WebAssembly (abbreviated Wasm) is a binary instruction format [...] designed as a portable compilation target for programming languages, enabling deployment on the web for client and server applications'' \autocite{WebAssembly}. Although it is not a programming language itself, a runtime environment that can run Wasm applications would be able to run applications written in a variety of languages.

\subsection{IAM standards}
To gain a deeper understanding of the problem space, it is important to briefly discuss the dominant IAM technologies. According to \textcite{Naik2016}, the most established identity protocols in the cloud computing industry are the following three: Security Assertion Markup Language (SAML), OAuth, and OpenID Connect (OIDC). 

\subsubsection{SAML}
Security Assertion Markup Language (SAML) and its associated protocols form an open standard for communicating information, ``packaged'' in an \emph{assertion}, about a subject, such as a user \autocite{Kemp2005}. SAML assertions, provided by an \emph{identity provider}, may represent authentication information, authorization decisions, or attributes associated with the subject. A \emph{service provider} can then use these assertions for access control and identification.

The use of SAML 2.0 can support single sign-on (SSO) of browsers, as specified by the Web Browser SSO Profile \autocite{Hughes2005}. When an unauthenticated HTTP user agent, in this case a browser, requests a secured resource from a service provider, the service provider issues a SAML authentication request to be delivered to the identity provider by the user agent. The identity provider identifies the subject, and issues a SAML response to be delivered to the service provider by the user agent. The service provider can then establish a security context for the subject and return the requested resource. 

\subsubsection{OAuth}
``The OAuth 2.0 authorization framework enables a third-party
application to obtain limited access to an HTTP service, either on
behalf of a resource owner [...], or [...] on its own behalf.'' \autocite{Hardt2023}


A typical OAuth flow consists of a \emph{client} (e.g. an application) requesting authorization from a \emph{resource owner}. When the resource owner issues an \emph{authorization grant} to the client, it may then exchange that authorization grant for an \emph{access token} from an authorization server. Using that access token, it may then access a protected resource, owned by the resource owner, on the resource server. \autocite{Hardt2023}
\begin{figure}[h]
\begin{scriptsize}
\begin{verbatim} 
+--------+                               +---------------+
|        |--(A)- Authorization Request ->|   Resource    |
|        |                               |     Owner     |
|        |<-(B)-- Authorization Grant ---|               |
|        |                               +---------------+
|        |
|        |                               +---------------+
|        |--(C)-- Authorization Grant -->| Authorization |
| Client |                               |     Server    |
|        |<-(D)----- Access Token -------|               |
|        |                               +---------------+
|        |
|        |                               +---------------+
|        |--(E)----- Access Token ------>|    Resource   |
|        |                               |     Server    |
|        |<-(F)--- Protected Resource ---|               |
+--------+                               +---------------+
\end{verbatim}
\end{scriptsize}
\caption{Abstract Protocol Flow \autocite{Hardt2023}}
\end{figure}

A JSON Web Token (JWT) may be used as both as OAuth 2.0 authentication grants and tokens \autocite{Jones2015a, Bertocci2021}. It provides a secure way to communicate information about a subject (referred to as claims), encoded as a JSON object which may be signed and/or encrypted. \autocite{Jones2015}

\subsubsection{OpenID Connect}
As detailed by its specification, OpenID Connect (OIDC) is ``a simple identity layer on top of the OAuth 2.0 protocol''. Information about the end user is conveyed in claims, as part of a JWT called an ID token. This information can then be used by a client to authenticate the user. \autocite{Sakimura2014} 

% Voor literatuurverwijzingen zijn er twee belangrijke commando's:
% \autocite{KEY} => (Auteur, jaartal) Gebruik dit als de naam van de auteur
%   geen onderdeel is van de zin.
% \textcite{KEY} => Auteur (jaartal)  Gebruik dit als de auteursnaam wel een
%   functie heeft in de zin (bv. ``Uit onderzoek door Doll & Hill (1954) bleek
%   ...'')

%---------- Methodologie ------------------------------------------------------
\section{Methodology}%
\label{sec:methodology}

The research will be divided into 6 phases, the first of which is a literature review. Literature research will be performed to explore existing approaches, find expert opinions, and gain a deeper understanding of the state of the art. The result of this literature search can be found in section 2 of this research proposal.

Phase 2 consists of creating a list of minimal requirements for a suitable language choice. Possible types of requirements could include ease of use, security implications, performance, and suitability for IAM-related tasks. A concrete list of measurable requirements will be assembled in consultation with thesis stakeholders and employees at TrustBuilder. The MoSCoW method will be used to order these requirements by importance, and each requirement will be marked as functional or non-functional.

Phase 3 is to create a long list of options, checked against the previously listed requirements. The literature review from phase 1 will be used to build an exhaustive list of suitable language choices. Popular existing options, as well as possibly lesser-known choices, will be considered.

Phase 4 aims to create a short list of the three most promising options, using a requirements summary table.

In phase 5, a proof of concept will be created to validate the suitability of each language in the short list. The proof of concept will consist of a simple HTTP server for each chosen language, which will allow running code that handles HTTP requests in the specified language. Three scripts will be written for each language, and will handle three distinct needs. The exact functionality of these three types of scripts will be decided based on TrustBuilder's typical customization requirements. This proof of concept will help further solidify and compare each language's suitability for customizing and extending IAM systems, and will provide deeper insight in their tooling and ease of use for these purposes.

Phase 6 will consist of drawing conclusions out of the research and propose next steps. The results and findings from the previous phases will be analyzed, and limitations and areas for future research will be identified.

Throughout this entire process, a thesis will be written that documents the research, findings, and proposed solutions.


%---------- Verwachte resultaten ----------------------------------------------
\section{Expected result, conclusion}%
\label{sec:expected_results}

Based on the findings in the literature review, JavaScript/TypeScript is expected to be the ideal language for writing extensions for IAM systems. It is currently the most popular programming language, and is already widely used for the extension of platforms, both in the general web development space as for IAM-related tasks. Being the most popular language, it has a rich package ecosystem, is actively being developed, and has a range of possible runtime environments. Having been specifically created for the web, with new standardized web APIs still being added, it can easily handle OAuth and SAML requirements, as both technologies were designed on top of existing web standards. JavaScript is already an excellent language choice for these purposes, and its continued innovation cements its place for years to come.

As many languages can be compiled to JavaScript or WASM now, an extension system built on top of a JavaScript and WASM engine would also enable the usage of other languages, as technologies and requirements continue to evolve. Python remains a popular and easy-to-use language for scripting, and Rust is frequently used to safely write critical code. A system that enables this level of versatility would provide a future-proof platform for mission-critical customization scripts.



\printbibliography[heading=bibintoc]

\end{document}